\documentclass[msc,pdftex]{Estilos/coppe}
%Opções principais
% msc - Mestrado
% dscexam - Qualificação de Doutorado
% dsc - Doutorado
%doublespacing HAHHAHAHAHAH

%%%%%%%%%%%%%%%%%%%%%%%%%%%%%%%%%%%%%
%
% SE FOR EM PORTUGUES, USAR BABEL
%\usepackage[brazil]{babel}
%
%%%%%%%%%%%%%%%%%%%%%%%%%%%%%%%%%%%%%




% GErencie todos os pacotes usados aqui
\usepackage[utf8x]{inputenc}
\usepackage[T1]{fontenc}


\usepackage{enumitem}
\usepackage{multicol}


\usepackage{verbatim}
\usepackage{amsmath,amssymb}
\usepackage{relsize}
\usepackage{graphicx}
\usepackage{courier}
\usepackage{subfig}
\usepackage{caption}
\usepackage[hyphens]{url}
\usepackage{hyperref}
\usepackage{lmodern}
\usepackage{rotating}
\usepackage{ucs}
\usepackage[ruled,vlined,linesnumbered]{algorithm2e}
\usepackage{multirow}
%\usepackage{array}
\usepackage[bottom]{footmisc}

\usepackage{float}
% Resolva todas as questões de letras e grafias aqui

\PrerenderUnicode{ě} 
% coloque todos os pacotes ligados 
% a tikz e pgfplots aqui

\usepackage{tikz}
\usepackage{pgfplots}
\usepgflibrary{fpu}
\usetikzlibrary{shapes.multipart, mindmap, arrows}
\pgfplotsset{compat=1.14}


% TODOS OS COMANDOS NOVOS PODEM FICAR AQUI

%\renewcommand\thesubfigure{ (\alph{subfigure})}

%apud citation style
\newcommand{\apudp}[2]{(\citeauthor{#1},\space\citeyear{#1},\space as cited in \citeauthor{#2},\space\citeyear{#2})}


%full centered cell
\newcommand{\specialcell}[2][c]{%
  \begin{tabular}[#1]{@{}c@{}}#2\end{tabular}}
  
%\addto{\captionsenglish}{\renewcommand{\bibname}{Referências Bibliográficas}}
%preventing widow and orphan lines
\widowpenalty=10000
\clubpenalty=10000
\postdisplaypenalty=10000



\makelosymbols
\makeloabbreviations


\begin{document}


%%%%%%%%%%%%%%%%%%%%%%
% Isso define título e também
% ficha bibliográfica e páginas 
% iniciais

\title{Model for Thesis or Dissertation at COPPE/PESC}
\foreigntitle{Modelo para Tese ou Dissertação no PESC/COPPE}
\author{Geraldo}{Xexéo}
\advisor{Prof.}{Geraldo Bonorino}{Xexéo}{D.Sc.}
%\advisor{Prof.}{Nome do Segundo Orientador}{Sobrenome}{Ph.D.}
%\advisor{Prof.}{Nome do Terceiro Orientador}{Sobrenome}{D.Sc.}
	
\examiner{Prof.}{Geraldo Bonorino Xexéo}{D.Sc.}
\examiner{Prof.}{Nome do Quarto Examinador Sobrenome}{Ph.D.}
\examiner{Prof.}{Nome do Quinto Examinador Sobrenome}{Ph.D.}
% Aqui use apenas a sigla e tudo será gerado
\department{PESC}
% aqui use números apenas e o texto será gerado
\date{06}{2018}
	
\keyword{keyword1}
\keyword{keyword2}


\maketitle

\frontmatter
  
% INCLUDES PULAM PÁGINA
% E NÃO USAM .TEX NO FINAL
  
\dedication{Dedicated to someone \\ that you love \\ or admire.}


\chapter*{Agradecimentos}

É muito importante agradecer à todos que te ajudaram e apoiaram


\begin{abstract}

Você precisa de um resumo em português

\end{abstract}


\begin{foreignabstract}

You need an abstract in a foreign language

\end{foreignabstract}



\tableofcontents
\listoffigures
\listoftables
\printlosymbols
\printloabbreviations

\mainmatter
	

\chapter{Introduction}

This chapter presents $\ldots$.

It contains the motivations as well as a broad spectrum view of my work, also introducing core concepts that will be used in this dissertation.


\section{Motivation}

This section should give the motivation for the thesis.

It is interesting to give an economic value to the work, citing numbers from industry, etc.
\input{Contents/cap1.Intro/question.tex}
\section{Structure of this work}

This section should describe all chapters.
\chapter{Literature Review}

This chapter should introduce the topic of the thesis. It should give a top-down view of the problem being solved, from the most general topic to the most specific.

The end of this chapter should map the related work should also map all the related work.


\chapter{Methodology}

This chapter should cover not only the methology, but all methods and tools used in the thesis.

It should cover the most common methods in a superficial way, pointing to the literature, however, more specific or esoteric methods should be detailed.


\chapter{Proposal}
\label{chap:proposal}

This chapter should make a theoretical, mathematical or abstract proposal to solve the problem.

It could have theorems, software requirementes, software architectures, etc.
\chapter{Implementation}
\label{chap:implementation}
This chapter should detail the implementation.
\chapter{Evaluation}

This chapter should evaluate the solution proposed in \ref{chap:proposal} and implemented in \ref{chap:implementation}.

It should describe experiments, qualitative or quantitative, test cases.


\chapter{Conclusion}

The conclusion must, at least:
\begin{itemize}
    \item review the goals and how they were accomplished,
    \item describe the abstract results,
    \item describe the artifacts created, and
    \item cite the articles.
\end{itemize}
	
\backmatter



\nocite{*}
	
\bibliographystyle{Estilos/coppe-plain}
\bibliography{bibli}

% aqui entram os apêndices

\appendix
	
\chapter{Podem ser necessários apêndices}

Lembrem que o apêndice é normalmente algo que se faz referência no corpo da tese é por seu tamanho não fica muito bom quando lá colocado.

Exemplos de apêndices são código, formulários, tabelas de resultado de experimentos completas, etc...



\end{document}
