\documentclass[msc,pdftex]{Estilos/coppe}
%Opções principais
% msc - Mestrado
% dscexam - Qualificação de Doutorado
% dsc - Doutorado
%doublespacing HAHHAHAHAHAH

%%%%%%%%%%%%%%%%%%%%%%%%%%%%%%%%%%%%%
%
% SE FOR EM PORTUGUES, USAR BABEL
%\usepackage[brazil]{babel}
%
%%%%%%%%%%%%%%%%%%%%%%%%%%%%%%%%%%%%%




% GErencie todos os pacotes usados aqui
\usepackage[utf8x]{inputenc}
\usepackage[T1]{fontenc}


\usepackage{enumitem}
\usepackage{multicol}


\usepackage{verbatim}
\usepackage{amsmath,amssymb}
\usepackage{relsize}
\usepackage{graphicx}
\usepackage{courier}
\usepackage{subfig}
\usepackage{caption}
\usepackage[hyphens]{url}
\usepackage{hyperref}
\usepackage{lmodern}
\usepackage{rotating}
\usepackage{ucs}
\usepackage[ruled,vlined,linesnumbered]{algorithm2e}
\usepackage{multirow}
%\usepackage{array}
\usepackage[bottom]{footmisc}

\usepackage{float}
% Resolva todas as questões de letras e grafias aqui

\PrerenderUnicode{ě} 
% coloque todos os pacotes ligados 
% a tikz e pgfplots aqui

\usepackage{tikz}
\usepackage{pgfplots}
\usepgflibrary{fpu}
\usetikzlibrary{shapes.multipart, mindmap, arrows}
\pgfplotsset{compat=1.14}


% TODOS OS COMANDOS NOVOS PODEM FICAR AQUI

%\renewcommand\thesubfigure{ (\alph{subfigure})}

%apud citation style
\newcommand{\apudp}[2]{(\citeauthor{#1},\space\citeyear{#1},\space as cited in \citeauthor{#2},\space\citeyear{#2})}


%full centered cell
\newcommand{\specialcell}[2][c]{%
  \begin{tabular}[#1]{@{}c@{}}#2\end{tabular}}
  
%\addto{\captionsenglish}{\renewcommand{\bibname}{Referências Bibliográficas}}
%preventing widow and orphan lines
\widowpenalty=10000
\clubpenalty=10000
\postdisplaypenalty=10000
% algum limite para a numeração ou fica bizarro
\setcounter{secnumdepth}{4} %numbering subsubsections
\setcounter{tocdepth}{4} %subsubsections in table of contents


\makelosymbols
\makeloabbreviations


\begin{document}


%%%%%%%%%%%%%%%%%%%%%%
% Isso define título e também
% ficha bibliográfica e páginas 
% iniciais

\title{Model for Thesis or Dissertation at COPPE/PESC}
\foreigntitle{Modelo para Tese ou Dissertação no PESC/COPPE}
\author{Geraldo}{Xexéo}
\advisor{Prof.}{Geraldo Bonorino}{Xexéo}{D.Sc.}
%\advisor{Prof.}{Nome do Segundo Orientador}{Sobrenome}{Ph.D.}
%\advisor{Prof.}{Nome do Terceiro Orientador}{Sobrenome}{D.Sc.}
	
\examiner{Prof.}{Geraldo Bonorino Xexéo}{D.Sc.}
\examiner{Prof.}{Nome do Quarto Examinador Sobrenome}{Ph.D.}
\examiner{Prof.}{Nome do Quinto Examinador Sobrenome}{Ph.D.}
% Aqui use apenas a sigla e tudo será gerado
\department{PESC}
% aqui use números apenas e o texto será gerado
\date{06}{2018}
	
\keyword{keyword1}
\keyword{keyword2}


\maketitle

\frontmatter
  
% INCLUDES PULAM PÁGINA
% E NÃO USAM .TEX NO FINAL
  
\include{Contents/cap0.frontmatter/dedication}
\include{Contents/cap0.frontmatter/thanks}
\include{Contents/cap0.frontmatter/resumo}
\include{Contents/cap0.frontmatter/foreignAbstract}

\tableofcontents
\listoffigures
\listoftables
\printlosymbols
\printloabbreviations

\mainmatter
	

\include{Contents/cap1.Intro/Intro}
\include{Contents/cap2.revision/Revision}
\include{Contents/cap3.methodology/Methodology}
\include{Contents/cap4.proposal/Proposal}
\include{Contents/cap5.implement/Implementation}
\include{Contents/cap6.evaluation/Evaluation}
\include{Contents/cap7.conclusion/Conclusion}
	
\backmatter



\nocite{*}
	
\bibliographystyle{Estilos/coppe-plain}
\bibliography{bibli}

% aqui entram os apêndices

\appendix
	
\chapter{Podem ser necessários apêndices}

Lembrem que o apêndice é normalmente algo que se faz referência no corpo da tese é por seu tamanho não fica muito bom quando lá colocado.

Exemplos de apêndices são código, formulários, tabelas de resultado de experimentos completas, etc...



\end{document}
